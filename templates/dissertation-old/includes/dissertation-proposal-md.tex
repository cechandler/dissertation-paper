\section{Introduction}\label{introduction}

\setlength{\parindent}{0.4in} Born and raised in England, Jonathan Harvey is a multi-faceted composer whose music contains many qualities associated with British composers: a propensity towards elegance, melodicism, nuanced orchestral color, and rigorous compositional craft. Yet his musical interests and stylistic evolution depart remarkably from the other British composers of his generation. Serialism, spectralism, electronic music, and spirituality are all prominent threads found in his music, and the combination of these elements give rise to the uniquely identifiable and diverse fabric of Harvey's music. His compositional output maintains a cohesive vision, one informed by his spirituality and is founded upon strong formal architecture.

Early in his compositional life he was attracted to serialism, and from 1969 to 1970, he attended Princeton University under auspices of a Harkness Fellowship to study with composer Milton Babbitt. Beginning in the late 1970s, Harvey's growing interest in timbre and extended instrumental techniques lead him to begin incorporating microtonal harmonies based on the spectra of harmonic and inharmonic sounds, a decidedly mainland European approach to harmony. He began including electronics and other elements of music technology as a direct result of this trajectory, and he made use of them in works ranging in forces from solo instrument to large orchestra. These spectral and technological approaches to composition began in earnest with Harvey's close collaboration with IRCAM in the early 1980s, when he produced two of his most well-known works: \emph{Mortuos Plango, Vivos Voco} (1980) for eight-channel tape and \emph{Bhakti} (1982) for large ensemble and four-channel tape.

Harvey's eclectic compositional voice is inexorably connected to his spirituality, which serves as the aesthetic foundation upon which all his works are built. Harvey was raised a Christian, influenced by the works of Rudolf Steiner, attracted to Hinduism, and finally `largely' identified as a Buddhist from his late-30s onward. Harvey did not convert from one religion to another but rather drew from each to create a spiritual identity that was as multifaceted as his musical output. The mystical and religious sources that inspired his works were diverse: the texts of the Rig Veda,\footnote{\emph{Bhatki} (1980)} the poetry of Rabindranath Tagore,\footnote{\emph{Song Offerings} (1985) and \emph{White as Jasmine} (1999)} the Buddhist concept of emptiness,\footnote{\emph{Forms of Emptiness} (1986), \emph{Wheel of Emptiness} (1997), and \emph{Bird Concerto with Pianosong} (2001)} and Matthias Gr{ü}newald's depiction of Christ's crucifixion in his \emph{Isenheim Altarpiece},\footnote{\emph{Death of Light, Light of Death} (1998)} to name but a few.

Harvey's spirituality directly informed the compositional constraints he placed on his works, specifically in regard to technique and form. This is clear in the case of vocal and choral music where the texts he sets are mystical or spiritual in nature. In his instrumental and electroacoustic music, Harvey also viewed the compositional techniques he used to focus his process as being connected to his spirituality too. For instance, he frequently used symmetrical harmonic spaces (or fields) that not only serve as a technical foundation but also support spiritual imagery. ``I think symmetrical fields and harmonies are able to capture states of consciousness which are achieved in meditation or in samadhi or, as I say, to encourage them in people because of their self-contained floating nature. I felt there was a relationship between imagination and the technique here.''\footnote{Whittall, \emph{Jonathan Harvey}, 21. } The `self-contained floating nature' Harvey refers to is his belief that these harmonic fields constructed around a central pitch axis provide an alternative to bass-driven harmonic syntax.

While this description of his approach to harmonic spaces seems at odds with his use of microtonal spectra, where strong fundamental bass frequencies are colored by upper microtonal harmonics of varying amplitudes, the two approaches co-exist within the same piece and occasionally are presented simultaneously. Furthermore, Harvey draws different spiritual and mystical metaphors from his use of spectral and timbral techniques both in acoustic orchestration and in electroacoustic music.

\singlespacing

\begin{quote}
When you are dealing with spectral matters you become very aware, in the studio and elsewhere, of the dialectic between fusion and fission that sounds can be part of a whole. \ldots{}the fascination is in the hide-and-seek process where sounds which you took to be individual, highly characterized sounds, identities instruments, whatever, can hide themselves and blend so perfectly you can't see them any more. They come in and out of identity. \ldots{} And when this process occurs often enough, in and out, hidden and revealed, have a we kind of mystical essence of music which I think is very important\ldots{} For me, it is the very nature of reality itself that behind individuality one discerns a unity. And that is heard all the time in spectral music and that's the fascination of it.\footnote{Ibid., 28. }
\end{quote}

\doublespacing

In addition to his use of serialism, symmetrical harmonic spaces, and spectral harmonies on the musical surface, Harvey's spiritual identity has also guided his approach to formal architecture. Harvey and other scholars have written on the intersections between his spirituality and the formal organization of his music, and often .

Closing statement and transition to scope Fascination with associative sets and landscapes.

\section{Scope and Methodology}\label{scope-and-methodology}

The scope of this dissertation involves two aims. The primary aim is to perform a comparative formal analysis of two of Harvey's later works, \emph{Wheel of Emptiness} (1997) and \emph{Body Mandala} (2006). The analysis and discussion of these works will focus on their architecture, primarily the way in which Harvey articulates large scale form through highly associative and recurring musical materials that have clear sonic, contextual, and orchestrational relationships. A secondary aim of this dissertation is to contrast the form of these two works with \emph{Be(com)ing} (1979), one of his early mature works that also contains a cyclical form. \emph{Be(com)ing} is scored for B\(\flat\) clarinet and piano, much smaller forces than the other two works, but it is meticulously crafted from the standpoint of cyclical form. The comparison between these works will focus on how Harvey developed his approach to constructing what he believed were more memorable forms. Regarding this formal development process Harvey has said, ``I began to see that it was difficult to perceive the identity of such material and it was not strong enough to build form very clearly. I wanted something more memorable so that when it recurred after a long absence -- after so many minutes -- it would still be recognizable, therefore form is present. If you don't recognize it, there is no form.''\footnote{Ibid., 22. }

In this dissertation, I plan to discuss how it is an example of a piece that differs in memorability from the two later works, despite the noteworthy and inventive formal features.

In order to examine these recurring associative elements in each of the works, I will use the general framework and metalanguage for discussing musical segmentation and analysis as outlined in Dora Hanninen's \emph{A Theory of Musical Analysis: On Segmentation and Associative Organization}. Rather than functioning as a methodology guiding the analytical process, Hanninen's theory provides a solid foundation as an interpretive tool that, among other things, supports the segmentation of salient elements on the musical surface with precise language and a focus on association. Hanninen frames analytical segmentation by introducing three domains (sonic, contextual, and structural) and five levels (orientations, criteria, segments, associative sets, associative landscape). She describes a domain as ``a realm of musical activity, experience, and discourse about it, bounded by the sorts of musical phenomena or ideas under consideration''\footnote{Hanninen, \emph{A Theory of Music Analysis}, 5. }, which takes into account the importance of psychoacoustics, repetition, association, categorization, and theories of musical structure. The five levels act as a non-hierarchic ``chain of conceptual prerequisites''\footnote{Ibid., 9. }

Hanninen notes that during the analytical process not only of these domains need be active at the same time.

landscape

This dissertation will be organized in three chapters. The first chapter will contain an introduction to Jonathan Harvey's life and work, an overview of the history and features of \emph{Wheel of Emptiness} and \emph{Body Mandala}, and a discussion of the relevant terminology from Hanninen that will be used in chapter two. The second chapter will contain the analyses of \emph{Wheel of Emptiness}, \emph{Body Mandala}, and \emph{Be(com)ing}. For each piece, I will identify its salient segments and associative sets and establish the associative landscape which will serve as the basis for later comparisons. I will draw on the three domains and five levels introduced by Hanninen to identify these segments in order to the sonic and contextual criteria that support these segmentations and identify the aspects that change between recurring segments. The third and final chapter will focus on an interpretation of the analyses and a comparison of the associative landscapes. Additionally I will discuss Harvey's beliefs about the Buddhist concept of emptiness, how it is realized in these works, and touch on additional areas for future research.

Mention tool I have built to analyze equal addition compressed spectra. Give a figure snapshot of it.

\begin{figure}[htbp]
\centering
\includegraphics{./figures/harmonic-series.png}
\caption{Compressed Spectra Tool}
\end{figure}

Harvey further indicates in his program notes that his microtonal treatment of harmony is centered around ``a modulating sequence of equal addition compressed spectra,''\footnote{Harvey, ``\emph{Wheel of Emptiness}.'' }, a technique that he developed and began using in \emph{Advaya} (1994)\footnote{Downes, \emph{Jonathan Harvey}, 95. } for cello, electronic keyboard, and electronics.

\emph{Wheel of Emptiness} was commissioned by the Belgium based Ictus Ensemble and premiered in 1997. It is scored for a large ensemble of 16 musicians and electronics, and throughout it's 16-minute duration, Harvey attempts ``to reconcile flowing almost chaotic music and cool, discrete objects, which have no connection with each other, but which repeat in a repetitive pattern.''\footnote{Harvey, ``\emph{Wheel of Emptiness}.'' } \emph{Body Mandala} is the first in a trilogy of orchestral works commissioned by the BBC for the BBC Scottish Symphony Orchestra that deal with the Buddhist concept of ``purification''. It is followed by \emph{Speakings} (2008) and \emph{\ldots{}towards a Pure Land} (2005) with the order of the entire cycle exploring notions of purification in body, speech, and mind. \emph{Body Mandala} is a kind of sonic translation of the composer's experience visiting North India where he observed purification rituals in Tibetan Buddhist monasteries. These sonic experiences of Northern India are directly translated in Harvey's work through his orchestration. The material of the Tibetan \emph{tungchens} is articulated by the trumpets, horns, and trombones, while the material of the \emph{gelings} is reflected in the melismatic use of four oboes, and Tibetan cymbals are used extensively by the percussion.

While \emph{Wheel of Emptiness} and \emph{Body Mandala} differ in terms of instrumentation and inspiration, their titles both contain circular imagery and share common ground on several levels. Both works feature microtonal harmonies,

The exposition of each work involves the alternation between two types of music: microtonal harmonies based on harmonic and inharmonic spectra. The form of many of Harvey's works have a cyclical structure where melodies, gestures, and passages return with regularity, sometimes in altered form and sometimes repeated verbatim. The returning material is clearly recognizable through sonic or contextual cues that include motivic or thematic identity, timbral similarity, and orchestrational articulation.

``The famous low horns, \emph{tungchens}, the magnificently raucous 4-note oboes, \emph{gelings}, the distinctive rolmo cymbals - all these and more were played by the monks in deeply moving ceremonies full of lama dances, chanting and ritual actions. There is a fierce wildness about some of the purifications, as if great energy is needed to purge the bad ego-tendencies. But also great exhilaration is present. And calm. The body, when moved with chanting, begins to vibrate and warm at different chakra points and `sing' internally. As it were, `lit up' with sound.''\footnote{Harvey, ``\emph{Body Mandala}.'' }

\section{Summary of Research}\label{summary-of-research}

Many of Harvey's seminal works have been examined in detail: \emph{Mortuos Plango, Vivos Voco} (1980), \emph{Bhakti} (1982), \emph{Song Offerings} (1985), \emph{Madonna of Winter and Spring} (1986), and \emph{Ritual Melodies} (1990). With the exception of \emph{White as Jasmine} (1999), \emph{String Quartet No. 4} (2003), and \emph{Speakings} (2008), many of his later works remained unexamined. The period of the 1980s was an intensely fruitful time for Harvey, and his later works nonetheless show a composer of incredible technical, expressive, and intellectual vision and sophistication.
