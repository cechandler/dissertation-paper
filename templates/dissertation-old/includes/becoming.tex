\chapter{Becoming Analysis}
Jonathan Harvey is widely known as a composer whose style is difficult to categorize due to the wide variety of compositional techniques he employed. Throughout his career, his eclectic output includes purely acoustic and electronic works as well as mixed electroacoustic works, nearly all of which make use of both serial and spectral techniques. Harvey's use of serialism extends back to his earliest works, and from 1969 to 1970, he attended Princeton University under auspices of a Harkness Fellowship to study with Milton Babbitt. While serial thinking played a prominent role throughout his compositional life, his growing interest in timbre and extended instrumental techniques beginning in the late 1970s lead him to incorporate microtonal and spectral aspects in his works. Harvey's close collaboration and affiliation with IRCAM in the early 1980s solidified this facet of his work for the remainder of his life.

Harvey's \emph{Be(com)ing} is a work for clarinet and piano composed in 1979 and first performed in London in October of 1981 by the commissioners Julia Holmes and Julian Elloway. The piece is quite clearly serial, and while it is not spectral, the clarinetist and pianist perform a wide variety of traditional and extended timbral techniques over the course of its 15 minute duration. These techniques include producing air sounds, multiphonics, microtones, key clicks, slap tongue, and glissandi for the clarinetist and muted notes, scrapped strings, and placing a plasticine modelling strip on the lower strings for the pianist. This contrast of traditional playing versus these timbral techniques evokes a dramatic, colorful, and unpredictable sonic landscape.

In his program notes for \emph{Be(com)ing}, Harvey indicates that the title refers to the way in which ``the music changes quite dramatically, becomes something, with regard to its surface patterns and ways of producing sounds on clarinet and piano.''1 Harvey's clever use of parentheses in the title not only calls attention to the dichotomy between becoming and being but also perhaps suggests that one can be contained within the other. Becoming indicates something that is transforming, while being indicates something that is fixed- something that is in transition versus something that has arrived.

While these two appear irreconcilable, Harvey addresses their unification in his work through form and technique. He notes that while the surface of the music is highly varied ``the basic structure does not change but goes on repeating the harmonic and melodic patterns as before: the play of the relative world within the absolute world of ‘being.’ For both instruments, silence (or near silence) is always followed by soft, gentle music, which is always followed by energetic music. This threefold sequence occurs throughout, but clarinet and piano are not ‘in sync.’, each follows its cycles independently of the other.''2 With this in mind, it is clear then that the unchanging underlying structure is the aspect of being (something that is fixed), while the mercurial surface of the music and where it ultimately goes is the aspect of becoming (something that is in continuous transformation).

In this paper, I will focus my analysis of \emph{Be(com)ing} on its two primary features: the cyclical form and harmonic organization. I will shed light on the way in which the underlying structural elements contain many cyclical features through a detailed examination of several formal levels. I will also discuss how the work's cyclical organization is closely connected with the serial organization of the pitch material.

A close reading of Harvey's program note indicates that there are two types of cycles occurring throughout the piece. One is a background structural cycle guiding the formal and serial organization of the piece. The other is a middleground energy or character cycle that guides the relative intensity and activity of the music (ie. the near silence, gentle, and energetic characters Harvey mentions). These two cycles continuously occur, but both instruments progress through them independently and at separate rates of speed.

Despite the misaligned energy and structural cycles that this formal scheme produces, the work can be segmented into three large sections. I have based these sections on noteworthy form bearing passages where the duo temporarily aligns in character and dynamics despite being in different places with regard to their structural cycle. These three passages are: section A mm. 1-67, section B mm. 68-101, and section C mm. 102-112. Figure 1 details these three passages as well as the structural and energy cycles contained within them.

The form and cycle list in Figure 1 reveals this misaligned cycles that occur throughout the form. Section A contains the first, second, and third structural cycle for the clarinetist and the first and majority of the second for the pianist. Section B marks the clarinet's fourth cycle and the end of the pianist's second cycle and entire third cycle. During the beginning of this passage, the clarinet and piano are roughly aligned in terms of energy but become misaligned and then realigned during its course. Section C functions as a coda or closing statement. For the first time in the piece, the duo aligns structurally, primarily due to the fact that the clear serial and cyclical organization of the material breaks down. While the coda has fragments of earlier material, the music has become something different than what has preceded it. Since the coda is the most unique passages of the music, it will be examined in greater detail later on.

While there is much diversity within these large sectional groupings, the global trajectory of each involves music that builds from a soft to loud dynamic and grows more dense and active in texture, thus acting as a macrocosm of the much shorter energy cycles which share the same profile. The opening of the piece and the passage from measure 67 to 74 both begin quietly but have quite different harmonic rhythms. The material at the beginning rapidly moves through various row forms and hexachordal segments, while the passage beginning at mm. 67 is harmonically static and has a slow rate of change. The conclusions of these sections from mm. 61-66 and mm. 92-102 are much more active in harmonic rhythm and feature a particular emphasis on timbral techniques, particularly multiphonics, muted or stopped strings, and harmonic tremolos in the piano.

Taking closer look at the structural cycles of each instrument, a noticeable pattern emerges. After an initial cycle where the musical material is introduced, both the clarinet and piano explicitly return to and reinterpret this previous material several times. The clarinet completes a total of four cycles, while the pianist completes three. In Figure 1, the beginning of the cycles in the second column indicates the measure where previous material is revisited with the exception of the first cycle. During each cycle, the duo ``re-reads'' the previous material and makes slight variations by changing dynamic energy, rhythmic and metrical displacement, and inserting new material.

This new material is not inserted trivially. Over the course of the clarinetist's entire four cycles, each phrase is performed a total of three times. The same is true for the piano with the appropriate adjustment, three cycles are completed and each phrase is performed two times. It becomes apparent that if an instrument performs a cycle X number of times but each phrase is only heard X-1 number of times that each of the cycles must be missing some material. In fact this is the case with the duo's independent structural cycles; upon every cyclical return some phrases or segments are left out, only to return during the next cycle. Sometimes this missing material is simply missing, resting instead of being articulated. Other times, new material, with a particular focus on timbral techniques, is dispersed within these missing phrase segments.

Given the separate cycles that each instrument is working through independently, the misaligned cyclical nature of the material produces a mosaic like form on the small scale where each instrument has several opportunities to perform a solo passage due to its partner's ``missing'' phrase. At other times an instrument will reinterpret a cyclical passage with a different degree of energy from the original instance while the other instrument continues with material that does not react to the change in temperament. For instance, in measure 53, the clarinetist begins revisiting material from measure 9. The initial appearance in measure 9 is extremely soft at pianississimo, while the beginning of measure 53 quickly moves from pianissimo to fortissimo. At this moment, the pianist continues with its cyclical material very quietly (as opposed to the initial appearance which was mezzo forte) and does not join the clarinet in its new found dynamic energy.

After having given an overview of the two instrument's formal relationship together, I will now briefly examine the clarinet and piano cycles separately. The clarinet's cyclical structure can be broken down into the following five segments: 1) cycle 1: mm. 1-30, 2) cycle 2: mm. 31-52, 3) cycle 3: mm. 53-67, 4) cycle 4: mm. 68-101, and 5) coda: mm. 101-112. As mentioned before, each of the cycles is marked so because it is a clear return back to material established in the first cycle. Measure 80 of the fourth cycle and measure 36 of the second cycle correspond to measure 14, and measure 75 of the fourth cycle and measure 53 of the third cycle correspond to measure 9.

This seemingly patchwork set of associations is difficult to conceptualize as it appears in the timeline based score. Figure 2 (included as a separate PDF file) expresses the form of the clarinet part by cutting up the music and aligning each cycle vertically, a representation more conducive to the material. The music is still organized and read from left to right, but each system corresponds to a new cycle. The diagram stops at the coda since its lack of cyclicity and explicit returning of material stops, making the layered representation less useful. Upon exploring this diagram, it is evident that these cycles are nearly verbatim with regard to pitch order, rhythmic articulation, and contour. The variations primarily involve dynamics, timbral modifications, and the interspersed newer material in the ``missing'' phrases as described earlier.

Two clear examples of these ``missing'' phrases in the clarinet part occur from measures 58-61 in the third cycle and measures 68-74 in the fourth cycle. Measures 58-61 of the third cycle is normally where the clarinetist would revisit the material which began the second cycle and occurs during measures 14-17 in the first cycle. But since this phrase has already been presented twice and there is still a fourth cycle beyond the current one, Harvey omits the phrase and gives the clarinetist a chance to rest. A similar event happens in measures 68-74 where the earlier material in measures 23-30 and 45-52 would occur. Instead, the clarinetist alternates between resting and performing a very quiet and delicate multiphonic that is new material.

The cyclical form of the piano is constructed in much the same way that the clarinet material is. The piano part's structure can be segmented in the following way: 1) first cycle: mm. 1-37, 2) second cycle: mm. 38-74, 3) third cycle: mm. 75-101, and 4) coda, mm. 102-112. As indicated in Figure 1, these cyclic groups do not align neatly with the clarinet cycles save for the beginning of the coda. But like the clarinet's cyclical structure, each one of these formal segments beyond the first cycle corresponds to a ``re-reading'' of the previous material. Since the piano completes one fewer cycles than the clarinet, each phrase articulated by the pianist is presented twice over the course of the three cycles (versus the three appearances for the clarinet). Again these ``re-readings'' are very clear returns in terms of pitch order, rhythmic articulation, and contour, but given the piano's much larger registral space, there are additional adjustments made to returning phrases through octave displacement. Figure 3 (also included as a separate PDF file) presents the piano material in the same fashion as figure 2 for the clarinet, and these ``re-readings'' and changes in return can be clearly identified. As can be done for Figure 2, simply looking up and down vertically on this diagram, one can see each of the two appearances for each piano phrase.

A close examination of Figure 3 reveals a number of notable features about the piano. In general, the piano part is immensely dense and has a very wide registral bandwidth. For example in the very first phrase from measure 1-2, the pianist's active range is between C1 and F7. The large majority of phrases throughout the piece maintain this bandwidth size. In terms of texture, many phrases begin with the articulation of a single line melody spanning a large range only to gradually grow in density when a second rhythmically independent voice enters. Often these two voices will eventually fuse into dense chordal tremolos and agitated repetitions. Just as in the clarinet, there are many instances of ``missing'' phrases that are replaced by new material or left out on certain cycles.

Due to the fact that the pianist moves through pitch material at a much quicker rate than the clarinet, there are instances when series of row forms return in the same order as previous material but which are articulated in a much different way so as to differentiate them. The piano has not only a much larger registral range to work with but also the ability to play harmonies and present two rows simultaneously in each hands. Compare, for example, the pitch sequence of the opening four measures with the passage from measure 26-32. Each passage presents exacting the same row forms, but they are projected very differently through changes in octave displacement and rhythmic pacing.

The underlying cyclical structure that serves as a foundation for \emph{Be(com)ing} is also expressed through Harvey's harmonic language. The piece's twelve-tone row, its properties, and its surface realization also reflect the tight conservation of material evident throughout. The principle row of the piece, established by the clarinet in measures 1-2, is <98342A670B15>. This row is a combination of two 6-7[012678] all-combinatorial D-type hexachords, which are highly invariant and produce a row matrix with only six distinct hexachordal collections. In the row matrix (see Figure 4), there are twenty-four uniquely ordered prime and inversion forms, but each of the six unique hexachords has four different versions whose order positions are rearranged. For example, P9 contains <98342A670B15> and P3 contains <329A8401657B>. Taking the previous P9 hexachord as a base, the four resulting order positions found for each hexachord in the matrix are <012345>, <240513>, <052431>, and <230154>. Due to the intervallic properties of this hexachord, the order position rearrangements occur from tritone pairs exchanging places ({02}{14}{35} in the first two pairs just mentioned).

In terms of hexachordal collections, a single hexachord's generalized relationship to its other prime reorderings can be described as such: Pn = Pn+6 = R(Pn+3) = R(Pn-3). The same relationship holds true for inversion forms: In = In+6 = R(In+3) = R(In-3). Each of the hexachords in these four pairs contain the same members but are simply reordered as noted in the four order positions possibilities above. The relationship between prime and inversion forms is even more closely related as each prime form has an order position invariant inversion form. For instance by comparing P9 <98342A670B15> and I6 <670B1598342A>, one can see that the two hexachords have exchanged position. With this hexachordal exchange in mind, the generalized relationship between the hexachordal pairs can be described as Pn = In-3.

Figure 4 shows three duplicated row matrices with the six unique hexachords highlighted in different colors. These invariant hexachords mimic the recombinant and cyclical nature of the music's surface and enable Harvey to mix and match different hexachordal row segments to complete the aggregate. For instance, using the P9 and P3 example above, Harvey can present the first hexachord from P9 and the second hexachord from P3 and still complete the aggregate but do so with different order position presentations.

The harmonic profiles of the row in Figure 5 detail its intervallic properties and cyclical set class organization. The two [012678] hexachords are ordered inversionally with respect to one another, projecting a directed intervallic sequence of <11, 7, 1, 10, 8, 8, 1, 5, 11, 2, 4, 4>. The undirected intervallic sequence <1, 5, 1, 2, 4, 4, 1, 5, 1, 2, 4, 4> highlights the row's cyclical and limited intervallic properties, containing no ic-3 or ic-6. This is not surprising since the [012678] hexachord contains no ic-3s on its own, and the inclusion of a ic-6 would render combinatoriality impossible. Set class cardinalities beyond the intervallic level also contain the same cyclical quality characterized by the two sequences of <1, 5, 1, 2, 4, 4>. For instance, at the trichord level, the row produces a sequence of 3-5[016], 3-4[015], 3-1[012], 3-8[026], 3-12[048], and 3-3[014] which occurs twice as the row is wrapped around back to the beginning. Each level of cardinality in the harmonic profile contains this cyclical property. These cyclical intervallic and set-class aspects of the row are a harmonic microcosm of the surface of the music. Just as the clarinetist and pianist continuously return to ``re-read'' passages, so does the various harmonic segmentations of the row.

A close examination of one passage, its cyclic variants, and the coda from \emph{Be(com)ing} will illustrate the way in which Harvey articulates the row and its hexachordal collections and how that presentation breaks down during the coda. Throughout the piece, the clarinet and piano's harmonic material is independent from one another, in the same way that their cyclical structures are independent. Presentations of the row and hexachordal segments are projected within a single instrument and are never grouped between the two. As such, I will look at each instrument's material separately.

The passage from measures 12-17 is the first time in the piece where the two duo's musical material becomes energetic together. In measure 14, the clarinet phrases become subito mezzo-forte as well as more agitated and separated compared to the previous phrases featuring long legato lines. In measure 12, the texture of the piano's material becomes more contrapuntal as a second voice explicitly enters in the left hand. Over the course of the passage, both instruments' activity increase, and they conclude with wide leaping fortissimo gestures before returning to silence and quiet, breathy material.

Figure 6 gives a detailed harmonic segmentation of this passage. The clarinet articulates row forms P7, I6, and P8, while the piano articulates I6, I5, and I4. The I6 in the clarinet is only presented as a hexachordal fragment. The second half of this row is not present, thus making the membership of the hexachord also possible as the second half of P9. However, there are many instances of row form projections descending by a half step (most notably in the row forms at the beginning) which is the basis for calling this hexachord a member of I6 instead of P9. The presentation of these rows forms occurs cyclically with two cycles in the clarinet and three cycles in the piano. Comparing these row forms to one another, P7 contains the same two invariant hexachords as I4, while P8 contains the same invariant hexachords as I5. Combined with the shared presentation of I6, both the clarinet and piano project not only similar hexachordal collections but also pitch class orderings. Here again is a microcosm of the way the larger cyclical form underlying the total work filters down to the smallest level of the piece.

Each statement of this passage follows the exact same harmonic structure as presented in Figure 6, but the several versions of this passage heard throughout do not coincide with one another again. The ``re-readings'' of this passage in the clarinet occur in mm. 13-17 (first cycle), mm. 35-39 (second cycle), mm. 57 (third cycle), and mm. 79-83 (fourth cycle). Returning briefly to the cyclic formal presentation of the clarinet part in Figure 2, when measures 35-39 are aligned with 13-17, the second cycle is missing the initial phrase <45A9B36> which corresponds to measure 13. The missing measure in the second cycle is found at the end of a phrase in the fourth cycle in measure 57. These two portions that became separated reappear joined in the fourth cycle. The second version of this material in the piano part does not align with any of the clarinet's ``re-readings.'' The first passage of course appears in measures 12-17 but the second passage is found from measures 48-53, directly between the second and third appearances of the passage in the clarinet.

Despite the unchanging harmonic segmentation of each statement, there are variations in terms of dynamic and timbral articulation as well as metrical and rhythmic displacement. The three instances of measure 13 in the clarinet (mm. 13, 57, and 80-1) are all uniquely performed. The first features a quiet, pianissimo, and legato articulation of a <45A9B36> phrase. The second version is highly dramatized with strong, staccato accents at a fortissimo dynamic level and interspersed with a piercing ``teeth on reed'' timbral technique. The third version returns to a quiet dynamic but employs a breathy tone quality as well as a multiphonic to articulate the D# pitch class. The pianist's material is also varied in terms of dynamic character and octave displacement. The first version of the passage continuously moves between a piano and forte dynamic, while the second version remains pianissimo for the entire duration.

Similarly, metrical and rhythmic displacements vary with each instance of this phrase. The first version is articulated as quarter note triplet plus a quarter note quintuplet, while the later two ``re-readings'' do not follow this rhythmic articulation. Metrically this phrase is also displaced each time; the first occurs on the downbeat, the second on a sixteenth note upbeat, and the third on the downbeat of beat three. Regarding the piano material, the second passage is consistently displaced by three quarter notes when compared to the first passage. While these metrical displacements may not actually be heard due to the context and general avoidance of periodic rhythmic structure, it is noteworthy as a musical dimension that is varied for returning material.

The harmonic, rhythmic, and cyclical organization of this passage is indicative of how the entire piece is technically constructed. Row forms and hexachordal segmentations continuously cycle, change, and then cycle, and on a larger sectional scale, entire passages are also revisited cyclically. Now that I have outlined a general framework of the piece's cyclical properties, I will briefly describe the coda where this technical structure comes to a halt.

The last ten measures of the piece function as a coda where the clarinet and piano finally structurally align both in terms of dynamic energy and lack of serial organization. Initially, the pianist begins the section in measure 102 with material that corresponds directly with the same music in measure 34. The music between mm. 103 and 106 roughly ``re-reads'' the material from measure 34-36, but due to its lack of clear return compared to the other previous cycles, I hesitate to call it the ending of the third cycle. Instead I view the haziness of its ``re-reading'' as the gradual breakdown of cyclicity. After this somewhat ``re-read'' material ends in measure 106, the remaining music until the ending is almost stuck on quick figures that repeat several times, for example the <179> and <378> fragments in mm. 107 and 108. The piano's left hand material at this point is slightly reminiscent of the left hand music in mm. 60-61 and 98-100. However, it differs a bit in harmonic content and moving at a quick pace than the previous passages.

While the pianist's music is at drastically different registral poles (the lowest octave in the left hand and highest octave in the left hand), the clarinet remains fixated on a much narrower register between C4 and F#5. Like the piano, however, the harmonic content of the clarinet's coda is lacking in serial organization but containing fragments reminiscent of earlier music. The persistent re-articulation of the F#5 upper registral boundary throughout the passage from mm. 102-112 recalls earlier F#5 gestures with an agitated character, such as those in mm. 28, 30, 44, 50, 66, 95, and 97-98. Additionally the character, contour, and rhythm of the gesture on the downbeat of measure 106 is the same as the gesture in measure 17 and measure 83. This small gesture is one of the only fragments that is not revisited three times during the clarinetist's earlier cyclic material (compared the gestures in Figure 2).

After having detailed the piece's tightly organized cyclical and serial structures that are expressed in many layers throughout the piece, what can be discerned from its absence at the end? My interpretation of this element of \emph{Be(com)ing} relates to Harvey's spirituality and generally mystical nature. As a deeply spiritual person, Harvey often drew inspiration for his works from mystical or religious sources. His spiritual identity was as multifaceted as his musical output, touching on aspects of Christianity, Hinduism, and Buddhism. One needs only to look at the titles of a list of his works for a clear indication of this: Inner Light, Bhakti, The Madonna of Winter and Spring, Wheels of Emptiness, Lotuses, and Body Mandala. While he was raised a Christian, Harvey became particularly interested in Hinduism and identified as a Buddhism from about his late-30s onward. He even went so far as to write an opera in 2007 (Wagner Dream) about the last day and death of Richard Wagner where he meets the Buddha and is given the chance to see the Buddhist opera he never had the opportunity to compose.

This is all to say that Harvey's strong interest in non-Western religions and those religions' belief in the cyclical nature of death and rebirth clearly resonates with the cyclical structure and its dissipation in \emph{Be(com)ing}. Hindu and Buddhist practitioners believe that one's ultimate goal in the physical world should be to liberate themselves from this cycle and to attain nirvana, which is essentially a spiritual state where the cycle of death and rebirth has been broken. Within this context, the title has deeper spiritual connotations beyond the literal meanings of the something in transformation versus something fixed. `Becoming' is perhaps the process of attaining enlightenment or nirvana whereas `being' is being stuck in the various cycles of existence. Whatever Harvey's intentions in this piece,  \emph{Be(com)ing} is meticulously crafted and compelling work whose technical foundations and poetic implications are deeply intertwined.
